\documentclass[12pt]{article}
%\usepackage{mathbbold}
\usepackage{ctex}
\usepackage{amsfonts,mathrsfs,amsthm,amssymb,amsmath}
\usepackage{multirow}
\usepackage{bbding}
\usepackage{graphicx,color}
\usepackage{rotating}
\usepackage[justification=justified,singlelinecheck=off]{caption}
\usepackage{bm}
\usepackage{epstopdf}
\usepackage[colorlinks,linkcolor=blue,urlcolor=blue,anchorcolor=blue,citecolor=blue]{hyperref}
\usepackage[noend]{algpseudocode}
\usepackage{algorithmicx,algorithm}
\usepackage{natbib}
\usepackage{longtable,booktabs}
\usepackage{geometry}
\usepackage{CJK}
\usepackage{comment}  % 加载 comment 包
%\usepackage{times}
%\usepackage{colortbl}
%\renewcommand{\theequation}{\thesection.\arabic{equation}}%for the form (1.1)(2.8)
\geometry{a4paper, left=2.5cm,right=2.5cm,top=2.5cm,bottom=2.5cm}

\linespread{1.4}

\renewcommand{\tablename}{\footnotesize\textbf{Table}}

%\theoremstyle{plain} \newtheorem{theorem}{Theorem}
%\theoremstyle{definition} \newtheorem{Defi}{Definition}
%\theoremstyle{remark} \newtheorem{Rem}{Remark}
%\theoremstyle{plain} \newtheorem{lemma}{Lemma}

\newtheorem{defi}{\sc Definition}[section]
\newtheorem{theorem}{\sc Theorem}[section]
\newtheorem{lemma}{\sc Lemma}[section]
\newtheorem{prop}{\sc Proposition}[section]
\newtheorem{coro}{\sc Corollary}[section]
\newtheorem{rema}{\sc Remark}[section]
%\newtheorem{exam}{\sc Example}[section]
\renewcommand{\theequation}{{\arabic{section}.\arabic{equation}}}

%the four below to generate roman number
\makeatletter
\newcommand{\rmnum}[1]{\romannumeral #1}
\newcommand{\Rmnum}[1]{\expandafter\@slowromancap\romannumeral #1@}
\newcommand{\M}{\mathcal{M}}
% 报错(Command \I already defined.LaTeX) \newcommand{\I}{\mathcal{I}}
\newcommand{\whb}{\widehat{\bm\beta}}

\makeatother

\renewcommand{\algorithmicrequire}{\textbf{Input:}}
\renewcommand{\algorithmicensure}{\textbf{Output:}}


% 报告主体部分
\begin{document}
\title{羽毛球球员表现的贝叶斯分析:\\ 胜率与多重因素的关系}
\date{}

\author{shenyuchen$^1$		\\		XXXX大学 XXXX学院}		% \\ 转义+换行
\maketitle

% 摘要
\begin{abstract}
统计决策理论与贝叶斯分析
统计决策理论与贝叶斯分析
统计决策理论与贝叶斯分析
统计决策理论与贝叶斯分析
\end{abstract}

% todo - 关键词?摘要
\textbf{关键词}: 贝叶斯;羽毛球;国际羽毛球联合会

\			% 换行,上面必须也要空一行

\section{背景介绍}
统计决策理论与贝叶斯分析
统计决策理论与贝叶斯分析
统计决策理论与贝叶斯分析
统计决策理论与贝叶斯分析

测试:
$$a+b=1$$

\section{数据收集与预处理}

\section{贝叶斯分析模型}

\section{结果分析与讨论}

\section{结论}

$$P(A|B)=\theta^2	\sum_{i = 0}^{\infty}	\{\theta (1-\theta ) \}^i = \frac{\theta ^2}{\theta ^2 - \theta +1}    $$


% 参考文献
\begin{thebibliography}{100}
	\expandafter\ifx\csname natexlab\endcsname\relax\def\natexlab#1{#1}\fi
	\expandafter\ifx\csname url\endcsname\relax
	\def\url#1{\texttt{#1}}\fi
	\expandafter\ifx\csname urlprefix\endcsname\relax\def\urlprefix{URL }\fi

\bibitem[{Battey et al.(2018)}]{battey2018distributed} 
Battey, H., Fan, J., Liu, H., Lu, J. and Zhu, Z., 2018. Distributed testing and estimation under sparse high dimensional models. \textit{Annals of statistics}, \textbf{46}, 1352–1382.

\bibitem[Chang et al.(2017)]{chang2017divide}
Chang, X., Lin, S.B. and Wang, Y., 2017. Divide and conquer local average regression. \textit{Electronic Journal of Statistics}, \textbf{11}, 1326-1350.

\end{thebibliography}


\end{document}


