\documentclass[12pt]{article}
%\usepackage{mathbbold}
\usepackage{ctex}
\usepackage{amsfonts,mathrsfs,amsthm,amssymb,amsmath}
\usepackage{multirow}
\usepackage{bbding}
\usepackage{graphicx,color}
\usepackage{rotating}
\usepackage[justification=justified,singlelinecheck=off]{caption}
\usepackage{bm}
\usepackage{epstopdf}
\usepackage[colorlinks,linkcolor=blue,urlcolor=blue,anchorcolor=blue,citecolor=blue]{hyperref}
\usepackage[noend]{algpseudocode}
\usepackage{algorithmicx,algorithm}
\usepackage{natbib}
\usepackage{longtable,booktabs}
\usepackage{geometry}
\usepackage{CJK}
\usepackage{comment}  % 加载 comment 包
%\usepackage{times}
%\usepackage{colortbl}
%\renewcommand{\theequation}{\thesection.\arabic{equation}}%for the form (1.1)(2.8)
\geometry{a4paper, left=2.5cm,right=2.5cm,top=2.5cm,bottom=2.5cm}

\linespread{1.4}

\renewcommand{\tablename}{\footnotesize\textbf{Table}}

%\theoremstyle{plain} \newtheorem{theorem}{Theorem}
%\theoremstyle{definition} \newtheorem{Defi}{Definition}
%\theoremstyle{remark} \newtheorem{Rem}{Remark}
%\theoremstyle{plain} \newtheorem{lemma}{Lemma}

\newtheorem{defi}{\sc Definition}[section]
\newtheorem{theorem}{\sc Theorem}[section]
\newtheorem{lemma}{\sc Lemma}[section]
\newtheorem{prop}{\sc Proposition}[section]
\newtheorem{coro}{\sc Corollary}[section]
\newtheorem{rema}{\sc Remark}[section]
%\newtheorem{exam}{\sc Example}[section]
\renewcommand{\theequation}{{\arabic{section}.\arabic{equation}}}

%the four below to generate roman number
\makeatletter
\newcommand{\rmnum}[1]{\romannumeral #1}
\newcommand{\Rmnum}[1]{\expandafter\@slowromancap\romannumeral #1@}
\newcommand{\M}{\mathcal{M}}
% 报错(Command \I already defined.LaTeX) \newcommand{\I}{\mathcal{I}}
\newcommand{\whb}{\widehat{\bm\beta}}

\makeatother

\renewcommand{\algorithmicrequire}{\textbf{Input:}}
\renewcommand{\algorithmicensure}{\textbf{Output:}}


% 报告主体部分
\begin{document}
\title{羽毛球球员表现的贝叶斯分析:\\ 胜率与多重因素的关系}		% TODO 使用贝叶斯推理分析和预测体育比赛结果
\date{}

% TODO
\author{shenyuchen$^1$		\\		XXXX大学 XXXX学院}		% \\ 转义+换行
\maketitle

% 摘要
\begin{abstract}
统计决策理论与贝叶斯分析
统计决策理论与贝叶斯分析
统计决策理论与贝叶斯分析
统计决策理论与贝叶斯分析
\end{abstract}

% TODO - 关键词?摘要
\textbf{关键词}: 贝叶斯;羽毛球;国际羽毛球联合会

\			% 换行,上面必须也要空一行

\section{引言(Introduction)}

\subsection{研究背景}
简要介绍羽毛球比赛的相关背景,为什么选择使用贝叶斯推理进行分析。

\subsection{研究目标}
阐述你希望通过贝叶斯推理预测比赛结果,探索相关因素。





% -----------------------
\section{数据收集与预处理(Data Collection and Preprocessing)}
\subsection{数据的来源}
"bwfbadminton" 是羽毛球世界联合会(Badminton World Federation,简称BWF)的官方网站的一部分。BWF 是羽毛球运动的全球管理机构,负责组织和管理国际羽毛球赛事,包括世界羽毛球锦标赛和世界羽毛球巡回赛等
。

在证据中提到的 "bwfworldtour.bwfbadminton.com " 是BWF世界巡回赛的官方网站,提供有关羽毛球赛事的信息和新闻更新
。此外,BWF 还通过其官方社交媒体账号发布重要信息和新闻,例如在2021年1月8日,BWF 官方网站发布了关于印尼羽毛球运动员违规事件的调查结果,并通过其官方Twitter账号 @bwfmedia 进行了发布
。

因此,"bwfbadminton" 可能指的是与BWF相关的羽毛球赛事或相关信息的网站,特别是与世界羽毛球巡回赛相关的部分。这个网站可能提供关于羽毛球比赛的最新动态、赛事结果、运动员信息以及相关技术分析等内容。


\subsection{处理数据}
python 爬虫(chatgpt) + 清洗数据
详细描述数据的特征及其重要性。


\section{贝叶斯推理模型(Bayesian Inference Model)}


解释贝叶斯定理和你如何应用它进行比赛结果预测。
详细描述模型的构建过程,包括先验概率、似然函数和后验概率的计算。

\subsection{马尔可夫链蒙特卡罗方法(MCMC)}
马尔可夫链蒙特卡罗(MCMC)是一种通过构造一个马尔可夫链来模拟目标分布的抽样方法。MCMC 方法可以应用于任何贝叶斯模型,特别是当模型较复杂,无法直接计算后验概率时。MCMC 可以帮助我们通过采样来近似后验分布。

应用步骤:
定义模型:和贝叶斯网络一样,首先需要定义模型和先验分布。此时模型通常包含复杂的参数,并且后验分布无法用解析的方式计算。
使用MCMC采样:使用 MCMC 算法(如吉布斯采样、Metropolis-Hastings 算法等)来从后验分布中进行采样。
估计后验分布:通过从后验分布中采样,计算参数的后验估计和不确定性。

应用案例:
在羽毛球比赛的预测中,如果比赛的预测模型包括多个变量,并且这些变量之间的关系比较复杂,传统的贝叶斯方法可能无法直接求解后验概率,此时可以采用 MCMC 方法来从后验分布中抽样,估计比赛结果的概率。

使用MCMC方法结合羽毛球预测,通常涉及以下步骤:

定义模型:首先,构建一个描述羽毛球比赛结果的贝叶斯模型。例如,模型可以考虑球员的历史表现、健康状况、场地条件等多个因素。

设定先验分布:为模型中的参数(如球员实力、比赛状态等)设定先验分布。可以根据历史数据或专家知识设置合理的先验。

采样过程:使用MCMC算法(如吉布斯采样或Metropolis-Hastings)从后验分布中进行采样。这一步用于估计模型参数的后验分布,处理复杂的参数空间。

推断与预测:根据采样结果,可以推断出比赛的胜负概率。例如,利用采样结果计算球员胜率,结合当前比赛的特定条件,预测比赛结果。

MCMC能够捕捉参数之间的复杂关系,提供不确定性度量,从而做出更为精确的预测。


$$P(A|B)=\theta^2	\sum_{i = 0}^{\infty}	\{\theta (1-\theta ) \}^i = \frac{\theta ^2}{\theta ^2 - \theta +1}    $$


\section{实验与结果(Experiment and Results)}
介绍如何进行模型训练、验证和评估。

\
展示模型的预测准确性、评估指标(如准确率、精度、召回率等)。


\section{讨论(Discussion)}
分析模型的优缺点,以及可能影响预测结果的因素(如球员状态、比赛场地等)。

\
讨论结果的实际意义以及如何改进模型。


\section{结论(Conclusion)}
总结研究结果,提出未来研究方向。
模型改进:可以尝试加入更多的特征,如比赛场地、球员的身体状况等,改进模型的预测效果。

\
其他机器学习方法:除了贝叶斯推理,还可以尝试使用其他机器学习方法(如决策树、随机森林等)进行对比分析。






% 参考文献
\begin{thebibliography}{100}
	\expandafter\ifx\csname natexlab\endcsname\relax\def\natexlab#1{#1}\fi
	\expandafter\ifx\csname url\endcsname\relax
	\def\url#1{\texttt{#1}}\fi
	\expandafter\ifx\csname urlprefix\endcsname\relax\def\urlprefix{URL }\fi

\bibitem[{Battey et al.(2018)}]{battey2018distributed} 
Battey, H., Fan, J., Liu, H., Lu, J. and Zhu, Z., 2018. Distributed testing and estimation under sparse high dimensional models. \textit{Annals of statistics}, \textbf{46}, 1352–1382.

\bibitem[Chang et al.(2017)]{chang2017divide}
Chang, X., Lin, S.B. and Wang, Y., 2017. Divide and conquer local average regression. \textit{Electronic Journal of Statistics}, \textbf{11}, 1326-1350.

\end{thebibliography}


\end{document}


